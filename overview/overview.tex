\documentclass[twoside,a4paper,11pt]{report}
\author{Jarrah Gosbell}
\title{Cyber Security Skill Development Overview}
\usepackage{longtable}
\usepackage{fancyhdr}
\pagestyle{fancy}
\renewcommand{\subsectionmark}[1]{\markright{#1}}
\renewcommand{\sectionmark}[1]{\markright{#1}}
\renewcommand{\chaptermark}[1]{%
	        \markboth{\thechapter.\ #1}{}}

\usepackage{graphicx}
\usepackage{setspace}
\usepackage{microtype}
\usepackage[T1]{fontenc}
\usepackage{lmodern}
\usepackage[section]{placeins} %Deal with floats being in the wrong place. 

\newcommand\frontmatter{%
    \cleardoublepage
  %\@mainmatterfalse
  \pagenumbering{roman}}

\newcommand\mainmatter{%
    \cleardoublepage
 % \@mainmattertrue
  \pagenumbering{arabic}}
\newenvironment{multipleabstract}[1]
  {\renewcommand{\abstractname}{#1}\newpage\begin{abstract}\thispagestyle{plain}}
  {\end{abstract}}

\makeatletter
\renewenvironment{abstract}{%
    \begin{center}%
      {\bfseries \vspace{5em}\abstractname\vspace{.5em}\vspace{\z@}}%
    \end{center}%
    %\quotation
}

%Sorting code floats. 
\usepackage{float}
\floatstyle{plain}
\interfootnotelinepenalty=10000
\setcounter{secnumdepth}{2}
\setcounter{tocdepth}{2}
\makeatletter
\usepackage{parskip}
\usepackage{hyperref}
\begin{document}
\frontmatter
\maketitle
\tableofcontents
	\listoftables
		\phantomsection
		\addcontentsline{toc}{chapter}{List of Tables}
\begin{multipleabstract}{Introduction}
	\addcontentsline{toc}{chapter}{Introduction}
	This document is the initial research for a cyber security training program for the CYSEC VECC---a small group of cyber security students at UNSW Canberra. 
	The proposed program will focus on the skills needed to be able to enter the field professionally.
	However, it's specific focus will be on those needed to compete in CTFs (cyber security ``capture the flag'' competitions), 
	which are the main testing activity of students studying in the field. 
	The goal of the training program is to give the reader an overview of the topics which need to be understood before beginning to compete, 
	while also giving detail on how many of the tools used work, for those desiring more depth. 

	This overview of challenges sets out the requirements of a number of CTFs, 
	giving detail on what each challenge requires before it can be completed. 
	These requirements will then be used in the creation of a training program, 
	ensuring that it contains the required skills for a wide set of CTFs. 

	This overview will use the following CTFs due to their wide range of topics and skill requirements:
	\begin{itemize}
		\item Cyber Security Challenge Australia 2015.
		\item Boston Key Party 2016.
		\item PlaidCTF 2015.
	\end{itemize}
	These CTFs will be broken down by the sections given by their creators, 
	with each containing a list and description of the skills required to complete the challenges within. 
	These categories will then be amalgamated into one table for each challenge. 
	This table will give both the number of times a particular skill was found in the challenge and a subjective weighting which considers how important that skill is to completing the task. 
	Using these two values, a measure of usefulness can be assigned to each skill. 
	This can then be used to determine how important any one skill is compared to the others within the challenge, and thus, how much it should be emphasised within a training program.

	At the end of the document is the overview of all challenges which have been analysed. 
	This overview contains the same content as the prior tables, but it has been condensed into a single reference with all duplicates removed and accounted for. 
	It is this final table which will be used in determining the content of the final training program. 
\end{multipleabstract}
\mainmatter
\chapter{CySCA 2015}
	\section{Overview of Challenge}
		The Cyber Security Challenge Australia (CySCA)\footnote{\url{https://cyberchallenge.com.au/}} is a national university level competition set by the Australian Signals Directorate 
		and sponsored by multiple IT, communication and finance companies. 
		In the 2015 competition, there were 64 teams that competed, each from Australian higher education institutions. 
		This competition has been increasing in both complexity and the number of competing teams each year since it's start in 2013. 
		The challenge has the goal of ``finding the next generation of Australian Cyber Security talent''.
		This makes this challenge a good starting point for compiling the requirements for a Cyber Security training program. 
	\section{Main Areas of Focus}
		This section will explain the skills required for the 2015 challenge. 
		Specifically, the requirements of all areas within the competition will be explained in the following sections:
		\begin{itemize}
			\item Corporate Network Penetration Test.
			\item Web Application Penetration Test. 
			\item Detect and Defend.
			\item Forensics.
			\item Hash Cracking and Cryptography.
			\item Programming.
			\item Reverse Engineering.
			\item Python Exploitation.
			\item Hack the Box. 
		\end{itemize}
		These sections were the areas of focus the competition, and correspond with some of the major areas of cyber security.
		Thus, these sections will be used to determine which skills are required for the competition in the following section. 
		\subsection{Corporate Network Penetration Test}
			This is a step by step Penetration Test of a corporate network. 
			It requires the member to start a DNS enumeration to work out the external systems on the network. 
			The member will then move through a number of systems, 
			until they gain full access to the remote domain and it's DNS server. 

			The official description of this section is as follows:\footnote{\url{https://github.com/CySCA/CySCA2015/tree/master/corporate\_network\_pentest}}
			\begin{quote}
				\textit{``The Chief Visionary Officer would like you to assess the security of the ECWI corporate network and explain the vulnerabilites located to non-technical staff. This will allow ECWI staff to provide sound advice to enterprises. Players should set their DNS server to 192.168.5.53''}
			\end{quote}

			Through an analysis of post challenge ``write ups'' and completing some of the individual tasks, 
			it was determined that the following skills and technical concepts are required to conduct the task: 
			\textit{DNS enumeration, Command Injection, HTTP Directory Traversal, MitM Attacks} and \textit{Reverse Engineering}. 
			Each of these is given in more detail below. 
			\begin{description}
				\item[DNS enumeration or zone transfer]
					This is the act of gathering data on internal systems based on the responses of external facing DNS servers. 
				\item[Command Injection]
					Using a system, inject commands for the underling shell. 
					This requires a good understanding of the shell of the running system. 
				\item[HTTP Directory Traversal]
					Forcing a web server to serve files that it should not have access to by moving along the directory structure.
				\item[Man-in-the-Middle HTTP attack]
					Intercepting traffic from server to client and either gathering data from that traffic or injecting code into it. 
				\item[C\# Reverse Engineering]
					Taking a program written and compiled into C\# and determining what it is trying to do and how. 
					This gives significant information on the system that was originally hidden. 
					However, this is a specific area of a larger topic. 
					An understanding of Reverse Engineering would be more appropriate to a training program as information about the specifics of C\# can be gathered during a CTF or attack. 
				\item[Hash Cracking]
					Taking the output of a one way algorithm and finding the original data. 
			\end{description}
		\subsection{Web Application Penetration Test}
			This is a penetration test of an Intranet site which is used for most of the communications of the business. 
			This site is also connected to the IT administrative equipment. 
			Succeeding in this challenge starts with a basic understanding of javascript, moving to session hijacking and other more complex web attacks. 
			
			The official description of this section is as follows:\footnote{\url{https://github.com/CySCA/CySCA2015/tree/master/web\_application\_pentest}}
			\begin{quote}
				\textit{
				``The initiative believe their intranet web site is secure due to them using all of the wellness advice that they provide to enterprises. However they are big believers in `trust but verify' and would like you to verify that the website is secure. You can find the website at http://www.ecwi.cysca''. }
			\end{quote}

			Through an analysis of post challenge ``write ups'' and completing some of the individual tasks, 
			it was determined that the following skills and technical concepts are required to conduct the task: 
			\textit{JavaScript, Obfuscation, Session Hijacking, URL Injection}, and \textit{Cryptography}. 
			Each of these is given in more detail below. 
			\begin{description}
				\item[JavaScript and information obfuscation]
					JavaScript is the language of the web. It is used to allow code to set up and act on a web page and should be able to be run on any client. 
					Often, information or the code itself is obfuscated into the website using JavaScript.
					Understanding how this is done will make finding that data far easier. 
				\item[Session hijacking]
					When a user connects to a website, it will create a cookie which can be used to check that the user has been authenticated. 
					This cookie should be accessible only to the person browsing the site. 
					However, session hijacking is a means of taking this cookie and pretending to be the original user. 
				\item[URL Injection]
					Knowing how a URL (and the underlying HTTP) works allows you to alter it to gain access that you shouldn't have. 
					This can be used to access other users data or to 
				\item[Cryptography]
					This is the process which is used to ensure that data transmissions are not readable to others eves dropping on the network. 
					%TODO: There may be more. 
			\end{description}
		\subsection{Detect and Defend}
			This section required the member to move from attack to defence. 
			It starts with cyber event reporting and moves to IPS tuning and exploit forensics. 
			The official description of this section is as follows:\footnote{\url{https://github.com/CySCA/CySCA2015/tree/master/detect\_and\_defend}}
			\begin{quote}
				\textit{``After being notifed by the ECWI network administrator, 
				the initiative took some network traffic dumps and recovered a malware sample. 
				However, they don't know how to discover if there is any maliciousness within.
				Once you have located the malicious data, 
				the initiative would like you to create STIX reports so they can share information 
				with enterprises to improve their cloud wellness.''}
			\end{quote}
			
			Through an analysis of post challenge ``write ups'' and completing some of the individual tasks, 
			it was determined that the following skills and technical concepts are required to conduct the task: 
			\textit{STIX Reporting, Network Forensics, Malware Analysis, Snort, Netflow} and \textit{PCAP Analysis}.
			Each of these is given in more detail below. 
			\begin{description}
				\item[STIX Reporting]
					Structured Threat Information Expression\footnote{\url{http://stixproject.github.io/}} is a standard (XML based) reporting scheme for cyber security incidents. 
					It is commonly used to communicate these events between organisations.
				\item[Network Forensics]
					When writing reports on attacks, one must be able to understand how the attack occurred. 
					Being able to determine what occurred during the attack is fundamental in creating useful threat intelligence. 
				\item[Malware analysis]
					This it the act of gathering information from captured malware. 
					It usually takes both static reverse engineering and sandboxed dynamic analysis. 
					One should also create hash based signatures so the malware can be found elsewhere. 
				\item[Snort] 
					This is a network IPS\footnote{\url{https://www.snort.org/}} that is used to ensure all traffic on the network follows a known set of rules. 
					Understanding how to configure these tools is a core part of network defence. 
				\item[Netflow]
					Netflow is a tool used to gather basic data on communications. 
					While it will not give content, it is useful in determining who accessed what part of the network and when. 
				\item[PCAP analysis] 
					Analysing packet captures is one of the more detailed ways of finding how an attack occurred and what information was released. 
					This is usually done using tools such as wireshark. 
			\end{description}
		\subsection{Forensics}
			This section involves both file and network forensics, as well as gathering information on previous attacks. 
			This section starts with a PCAP of an attack in which the country of origin was hidden and moves to Memory analysis to determine the means of the attack. 
			
			The official description of this section is as follows:\footnote{\url{https://github.com/CySCA/CySCA2015/tree/master/forensics}}
			\begin{quote}
				\textit{``Following on from a security report, the initiative would like you to use your forensics skills to get to the bottom of any wrongdoing (either internal or external) on their network. Using the following files, find out what is happening and who is behind it.'' }
			\end{quote}
			Through an analysis of post challenge ``write ups'' and completing some of the individual tasks, 
			it was determined that the following skills and technical concepts are required to conduct the task: 
			\textit{PCAP Forensics, DOXXING, CVEs, Attack Traffic Forensics, Disk/Memory Analysis} and \textit{Attack Persistence}.
			Each of these is given in more detail below. 
			\begin{description}
				\item[PCAP forensics]
					Reading and understanding packet captures in order to determine what occurred while the capture was being taken. 
					This can be used to gather location and identity data on the attacker and victim. 
				\item[DOXXING] 
					Given a small amount of information, determine the identity or location of a person or organisation.
				\item[CVEs and CVE to attack matching]
					The common vulnerabilities and exposures system is used to maintain a database of possible exploits. 
					This can be used to track down the attack vector that was used for exploiting a system. 
				\item[Attack traffic Forensics]
					Using a PCAP or NetFlow capture to determine what occurred during an attack. 
					This should give information on the systems affected and the attacker. 
				\item[Disk/memory Analysis]
					Gathering information from the disk or memory of a computer. 
					This can be done either live or in an image, but requires an understanding of tools and architecture.
				\item[Attack persistence]
					The method of making an attack self supporting. 
					This will allow the attacker to retain access to the system across reboots and other system changes. 
					However, it will leave a signature on the system that can be found. 
			\end{description}

		\subsection{Hash Cracking and Cryptography}
			This section required that the member break a number of hashes and determine weakness in home build cryptography solutions. 
			An understanding of John the Ripper or Hashcat would be a good starting point for this section. 
			However, a stronger understanding of cryptography is necessary in order to break the later challenges. 
			
			The official description of this section is as follows:\footnote{\url{https://github.com/CySCA/CySCA2015/tree/master/crypto\_and\_hash\_cracking}}
			\begin{quote}
				\textit{``The initiative provide wellness advice to enterprises and this includes information on choosing a good password. They would like you to see how resistant each of these passwords are to password recovery. The initiative do not have anyone skilled in Cryptography. They would like you to assess some of their secure crypto systems and solve a crypto problem discovered as part of a security response.''}
			\end{quote}

			Through an analysis of post challenge ``write ups'' and completing some of the individual tasks, 
			it was determined that the following skills and technical concepts are required to conduct the task: 
			\textit{Hashcat, John The Ripper, Network Forensics, Binary Operations, Hashing Theory, Cryptographic Flaws} and \textit{Programming}.
			Each of these is given in more detail below. 
			\begin{description}
				\item[Hashcat or John the Ripper] 
					These are common hash cracking tools, they both have the power to conduct significant attacks while being able to distribute over multiple machines. 
				\item[Network Forensics]
					Taking captured data or a live network and determining either it's make up or what was sent across it. 
				\item[Binary operations] 
					Working at the lowest level of data and code, understanding how the CPU itself interacts with this. 
					This is useful in both programming and cryptography. 
				\item[Hashing theory and cryptographic flaws]
					The one-way means of taking data and storing it in such a way that the hash cannot be used to get the original data. 
				\item[Programming]
					This is the act of telling a computer what to do and how to do it. 
					It is useful in a number of cyber security challenges and will be discussed further in the following section.
			\end{description}

		\subsection{Programming}
			This section is designed to test the members programming skill by giving them logic challenges which are not a part of normal programming. 
			It starts with word unscrambling, ending in mathematical calculation. 
			
			The official description of this section is as follows:\footnote{\url{https://github.com/CySCA/CySCA2015/tree/master/programming}}
			\begin{quote}
				\textit{``The initiative staff would like to show them how to `code'. They have a number of samples that they would like to see you solve in code to help them learn.''}
			\end{quote}

			Through an analysis of post challenge ``write ups'' and completing some of the individual tasks, 
			it was determined that the following skills and technical concepts are required to conduct the task: 
			\textit{Programming, Python, Networking, Logic, Trivia, Mathematics} and \textit{QR Codes}.
			Each of these is given in more detail below. 
			\begin{description}
				\item[Programming (Python)]
					Programming is the act of telling a computer what to do. 
					Python is a language that due to it's support and modules, as well as the ease of writing and error checking it has become the standard language in cyber security. 
				\item[Networking]
					This is the means of connecting two or more computers together. 
					At the base level it is the understanding of how these physical and logical connections work. 
					But at the higher programming level, it is an understanding of sockets and protocols. 
				\item[Logic]
					The structured thought process used to get to an answer. 
					This is especially important in programming as it forms the basis of how a computer works. 
				\item[Trivia and general knowledge] 
					These are items which while not necessary are helpful to understand in order to know what the challenge means. 
					For example, a number of older games were used in challenge hints, helping those who understood them to complete the challenge. 
				\item[Mathematics] 
					A general understanding of mathematics is useful in the programming of solutions to these problems. 
					High level mathematics is not required outside specialist fields such as cryptography, but a lower level understanding helps. 
				\item[QR Codes]
					These are digital codes similar (but more complex) to bar codes.
					The contain information as well as a means of restoring that information if the code is damaged. 
					This restorative process requires an understanding of mathematics as well as error correction. 
			\end{description}

		\subsection{Reverse Engineering}
			This section was designed to test the members reverse engineering and programming skills. 
			It starts with a semi-blind attack, with some source given, but not the rest of the program. 
			It then ends with a C programming and comparison challenge. 
			
			The official description of this section is as follows:\footnote{\url{https://github.com/CySCA/CySCA2015/tree/master/reverse\_engineering}}
			\begin{quote}
				\textit{``There are no reverse engineers in the initiative, however they have a number of systems with incomplete documentation and one backdoored version of their software. They would like you to determine the functionality to allow them to continue to use these services.''}
			\end{quote}

			Through an analysis of post challenge ``write ups'' and completing some of the individual tasks, 
			it was determined that the following skills and technical concepts are required to conduct the task: 
			\textit{Programming, Reverse Engineering, Logic, Code Testing} and \textit{HTTP}.
			Each of these is given in more detail below. 
			\begin{description}
				\item[Programming]
					The art of telling a computer what to do. 
				\item[Accessing networked programs] 
					This requires tools such as ``nc'' or socket programming in order to properly interface with the computer on the other end of the network. 
				\item[Blind Reverse Engineering] 
					This is the act of gathering information and algorithms from a system for which you are supplied no code or documentation. 
					You may be given a binary to attack, or you may have to logically reverse engineer it from the other side of a socket. 
				\item[Logic and code testing] 
					Logic is the key thought process required in understanding how a computer works. 
					Code testing requires you to interact with a program in every possible way, 
				\item[HTTP] 
					This is the protocol that the Web runs on. 
					Understanding it allows one to understand how many of the programs which have been created to interact on the Web will format their transmissions. 
					Furthermore, it is useful as it enables you to manually create requests and send them to servers that you are attempting to exploit. 
			\end{description}

		\subsection{Python Exploitation}
			In a similar way to C, Python has a number of flaws which can be exploited in poorly written code. 
			Some of these flaws are well known and obvious, but others are significantly harder to see. 
			This section starts with running a command without python builtins, ending with deobfuscating code and writing python shellcode. 

			The official description of this section is as follows:\footnote{\url{https://github.com/CySCA/CySCA2015/tree/master/python\_exploitation}}
			\begin{quote}
				\textit{``The initiative's customers use Python heavily, they have created a number of secure python systems for use by enterprises. Please verify that the cloud wellness is suitable on these systems.''}
			\end{quote}

			Through an analysis of post challenge ``write ups'' and completing some of the individual tasks, 
			it was determined that the following skills and technical concepts are required to conduct the task: 
			\textit{Python Programming, Python Builtins, Pickled Input, Obfuscation} and \textit{Python Shellcode}.
			Each of these is given in more detail below. 
			\begin{description}
				\item[Python coding]
					Programming in the python language is the main method of programming within cyber security. 
					Understanding this within the python exploitation section was fundamental to creating a working solution to the given problems. 
				\item[Python builtins]
					These are the language features that come with the base environment of python. 
					When no modules have been imported, python has only it's builtins. 
				\item[Pickled input]
					Pickling is a method of packing data for transmission or storage. 
					It is used to keep a set of data together over an action, but can also be exploited to give unexpected results. 
				\item[Obfuscation and Deobfuscation]
					This is an attempt at hiding information. 
					Rather than encrypting it and making the information cease to be there without the correct key, obfuscation simply hides it within the surrounding text or data. 
					Deobfuscation is the act of finding this information.
				\item[Python shellcode] 
					Python shellcode is code which is used to gain shell access to a computer running through a python program. 
			\end{description}

			
		\subsection{Hack the Box}
			This section requires the member to break into a webserver. 
			This server is running a number of services which make it vulnerable to attacks. 
			This attacks starts with a basic Heartbleed credential steal, ending with privilege escalation. 

			The official description of this section is as follows:\footnote{\url{https://github.com/CySCA/CySCA2015/tree/master/hackthebox1}\\\qquad\url{https://github.com/CySCA/CySCA2015/tree/master/hackthebox2}} 
			\begin{quote}
				\textit{``The initiative would like you to assess this machine on their network. They are concerned that it may have a number of high profile vulnerabilities.''}
			\end{quote}

			Through an analysis of post challenge ``write ups'' and completing some of the individual tasks, 
			it was determined that the following skills and technical concepts are required to conduct the task: 
			\textit{Metasploit, Nmap, Heartbleed, SSH, RSA Key Cracking, Privilege Escalation, NFS, SETUID} and \textit{Linux Privelages and Permissions}.
			Each of these is given in more detail below. 
			\begin{description}
				\item[Metasploit basics] 
					This is the exploitation framework used for most of the worlds current network exploitation. 
					It is based on modules written in Ruby which launch the attack and create a payload to execute on the system. 
				\item[Nmap]
					This is the tool used for scanning a computer or network and determining what is running on it. 
					It can be used on multiple targets and will give detailed information about the services running on the systems. 
				\item[Heartbleed]
					This is a vulnerability which occurred in April of 2014. 
					This allowed a malicious user to request far more information from the server than necessary in an attempt to get keys or other sensitive data stored in memory. 
					It is also recommended that competitors keep up to date with major vulnerabilities before entering a CTF as it is unlikely that this specific vulnerability will be used again.
				\item[SSH]
					The secure shell program used for gaining shell access to a networked computer. 
					This uses passwords or public key authentication in order to gain access. 
				\item[RSA key cracking] 
					This is the practice of taking a weak or known RSA key and testing it against other known keys in order to determine the other half of the key. 
				\item[Privilege escalation] 
					When gaining access to a system, often the exploit will not give you full system access. 
					Privilage escalation is the act of gaining a higher level of access to the system. 
				\item[NFS]
					Network File Share is one means of sharing files over a network. 
					Given it's configuration options, it can be difficult to create this in a secure manner. 
				\item[SETUID] 
					This is a bit within the permissions of a file which has it run as another user, no matter who executes it. 
					This is the basis of local privilege escalation. 
				\item[Linux Privileges and permissions] 
					Based on the user account and file ownership, Linux gives a given user a set level of access to the system. 
					Understanding this and the details which go along with it allows an attacker to have a better chance of finding the exploitable program within the system. 
			\end{description}
	\section{Breakdown of Requirements}
	 	The following is a breakdown of how many times a particular area was used in the challenge. 
		These areas are broken into Major Categories (in bold) and minor categories, with a count for how many times each arose. 
		There is then a weighting section which conveys how important a given section is to completing the whole challenge. 
		\begin{center}
			\begin{longtable}{| l | l | l |}
				\hline
				\textbf{Category} & \textbf{Number} & \textbf{Weight} \\ \hline 
				\endhead
				\multicolumn{3}{|r|}{{Continued on next page}} \\ \hline
				\endfoot
				\endlastfoot
				%---------------------------------------------------------------
				\textbf{Programming} & \textbf{7} & \textbf{1} \\ \hline
				\quad JavaScript & 1 & \\ \hline 
				\quad Obfuscation & 1 & \\ \hline
				\quad Python & 3 & \\ \hline 
				\qquad Shellcode & 1 & \\ \hline 
				\qquad Builtins & 1 & \\ \hline 
				\qquad Pickled input & 1 & \\ \hline 
				\quad Sockets & 2 & \\ \hline 
				\textbf{General Knowledge} & \textbf{6} & \textbf{0.5}\\ \hline 
				\quad CVE and CVE attack Matching & 1 & \\ \hline 
				\quad Binary Operations & 1 & \\ \hline 
				\quad Logic & 1 & \\ \hline 
				\quad Trivia, General knowledge & 1 & \\ \hline 
				\quad Mathematics & 1 & \\ \hline 
				\quad QR Codes & 1 & \\ \hline 
				\textbf{Forensics} & \textbf{6} & \textbf{0.7}\\ \hline
				\quad PCAP & 2 & \\ \hline
				\quad Network Forensics & 3 & \\ \hline 
				\quad DOXXING & 1 & \\ \hline 
				\quad Attack Traffic & 1 & \\ \hline 
				\quad Disk and Memory Analysis & 1 & \\ \hline 
				\textbf{Network Exploitation} & \textbf{4} & \textbf{0.8} \\ \hline 
				\quad Attack Persistence & 1 & \\ \hline 
				\quad Metasploit & 1 & \\ \hline 
				\quad Nmap & 1 & \\ \hline 
				\quad Heartbleed & 1 & \\ \hline
				\textbf{Reverse Engineering} & \textbf{4} & \textbf{0.6}\\ \hline 
				\quad C\# Reverse Engineering & 1 & \\ \hline
				\quad Blind Reverse Engineering & 1 & \\ \hline 
				\quad Code Testing & 1 & \\ \hline 
				\quad Malware Analysis & 2 & \\ \hline
				\textbf{Cryptography} & \textbf{3} & \textbf{0.8} \\ \hline 
				\quad Hash Cracking & 2 & \\ \hline
				\quad Cryptographic Flaws & 1 & \\ \hline 
				\textbf{Web Penetration} & \textbf{3} & \textbf{1} \\ \hline
				\quad HTTP Directory Traversal & 1 & \\ \hline
				\quad HTTP MITM & 1 & \\ \hline
				\quad Session Hijacking & 1 & \\ \hline
				\quad URL Injection & 1 & \\ \hline
				\textbf{Network Defence} & \textbf{3} & \textbf{0.6} \\ \hline
				\quad STIX Reporting & 2 & \\ \hline
				\quad Snort & 1 & \\ \hline
				\textbf{Networking} & \textbf{2} & \textbf{0.8}\\ \hline
				\quad DNS Enumeration and Zone Transfer & 1 & \\ \hline 
				\quad HTTP & 1 & \\ \hline 
				\textbf{Linux CLI} & \textbf{1} &\textbf{0.6} \\ \hline
				\quad Command Injection & 1 & \\ \hline 
				\noalign{\vskip 0.5cm}
				\caption{\label{tab:CySCABreakdown}Breakdown of CySCA Topics}
				\vspace{-1.3cm}
			\end{longtable}
		\end{center}
		Using both the weighting and the count, one can determine the most useful content to place into a training program. 
		While all of this content will be added, it will be ordered in the same way as it is here and weighted for importance. 

\chapter{Boston Key Party 2016}
	\section{Overview of Challenge}
		The Boston Key Party is a well known CTF that is run early in the year. 
		This CTF is set at a moderate level, requiring a large amount of background, but not being overly difficult. 
		It has been used for the last three years as a qualifier for the DEFCON CTF, 
		which is one of the most prestigious CTFs held each year at the DEFCON conference. 
		Furthermore, it is often seen by the higher level teams as a warm up CTF. 
		These attributes make this CTF a good starting point for second year members. 

		The team that runs this CTF are from a diverse background, with many contributing to other teams and large open source projects such as Radare2.
		Their site can be found at \href{https://web.archive.org/web/20160131055956/http://bostonkeyparty.net/}{bostonkeyparty.net}. 
		However, this site is only used for the actual CTF event, rather than information. 
		More information can be found at the \href{https://ctftime.org/event/252}{CTF Time page}.
	\section{Main Areas of Focus}
		This section will explain the skills required for the completion of the challenges within this CTF. 
		It will list the requirements for each section, defining them so that they can be implemented into the training program. 
		All components of the CTF will be explained in the following sections:\footnote{\url{https://ctftime.org/event/252/tasks/}}
		\begin{itemize}
			\item ``Pwn''
			\item Reversing
			\item Web
		\end{itemize}
		\subsection{Cryptography}
			This section is designed to test the members ability to determine cryptographical weaknesses. 
			It also requires an ability to program and understand code. 
			
			Through an analysis of post challenge ``write ups''\footnote{\url{https://github.com/ctfs/write-ups-2016/tree/master/boston-key-party-2016/crypto}}
			and completing some of the individual tasks, 
			it was determined that the following skills and technical concepts are required to conduct the task: 
			\textit{Programming, SHA Hashing, RSA Key Recovery, Elliptic Curve Encryption, ElGamal Signatures, Common RSA Attacks, DES Weak Keys} and \textit{Hash Collisions}.
			Each of these is given in more detail below. 
			\begin{description}
				\item[Programming]
					The act of giving a computer a set of actions to complete. 
				\item[SHA hashing] 
					The one way Secure Hashing Algorithm and it's possible weaknesses. 
				\item[RSA Key Recovery]
					Returning from a public key (which is weak due to poor generation) to a private key. 
				\item[Elliptic Curve Encryption and Attacks]
					This is a new encryption standard which was designed to replace AES. 
					It has a number of weaknesses that can be exploited if the implementation is poor. 
				\item[ElGamal Signatures]
					These are digital signatures which are used to sign documents as a certain person. 
					There are a number of security issues with this, mainly due to hash collisions. 
				\item[Common RSA Attacks] 
					RSA being a long serving standard has a number of common attacks, such as the weak random number generator found in some older versions of Debian. 
				\item[DES Weak Keys] 
					These are a small number of keys which cause the encryption and decryption modes of DES to act in the same or similar ways. 
					This makes attacking these keys significantly easier. 
				\item[Hash Collisions]
					By nature of taking something and turning it into a constant sized string, hashes have collisions. 
					Understanding how these collisions work and where they can be found, one can attack some forms of hash. 
					Specifically MD5 and SHA1 are vulnerable to this. 
			\end{description}
		\subsection{``Pwn''}
			This section requires the member to gain an understanding of how the target program works and attack the binary itself. 
			Skills such as shell coding and stack enumeration will be the focus of this section. 

			Through an analysis of post challenge ``write ups''\footnote{\url{https://github.com/ctfs/write-ups-2016/tree/master/boston-key-party-2016/pwn}}
			and completing some of the individual tasks, 
			it was determined that the following skills and technical concepts are required to conduct the task: 
			\textit{RIP Redirection, Common C Flaws, Smashed Stack Cleaning, Heap Overflow, Reverse Engineering} and \textit{libc Attacks}.
			Each of these is given in more detail below. 
			\begin{description}
				\item[RIP Redirection] 
					This is the process of altering the running of a program in such a way that the current instruction pointer now points at a section of memory controlled by the attacker. 
					It is often used in shellcode. 
				\item[Common C program flaws] 
					Issues such as string format vulnerabilities and buffer overflows are common amongst C (or similar) programs. 
					Understanding this allows for numerous new entry points into multiple programs. 
				\item[Cleaning after stack smashing] 
					Once you have smashed into the stack of a program, you will need to clean up the damage done. 
					This means that the program will continue running and the user will not be notified by a crashed program. 
				\item[Heap overflow] 
					While not as easy as stack based overflows, heap overflows can be used to gain the same level of access. 
					This is either through writing a program to the heap and then altering the RIP, or growing the heap to overwrite the stack. 
				\item[Reverse Engineering]
					This is the process of taking a compiled binary and reading through it to determine how it works and how to break it. 
				\item[libc attacks] 
					Many modern Linux programs use libc, rather than implementing their own calls or statically linking. 
					If these calls can be intercepted or redirected, the attacker can own the running process and have it do whatever they desire. 
			\end{description}
		\subsection{Reversing}
			This section requires that the member reverse engineer and understand the program that they are given. 
			This requires skill in a number of areas, not the least of which is programming and the use of a tool such as radare2. 

			Through an analysis of post challenge ``write ups''\footnote{\url{https://github.com/ctfs/write-ups-2016/tree/master/boston-key-party-2016/reversing}}
			and completing some of the individual tasks, 
			it was determined that the following skills and technical concepts are required to conduct the task: 
			\textit{Reverse Engineering, IDA64, Radare2, Programming, Assembly, Side Channel Attacks, ltrace} and \textit{strace}.
			Each of these is given in more detail below. 
			\begin{description}
				\item[Reverse engineering] 
					The act of reading through and determining how a program works. 
					This can also help in exploit development. 
				\item[Reverse engineering tools] 
					Tools such as GDB/PEDA, Radare2 and IDA are useful in reverse engineering and should be covered when training for it. 
				\item[Programming] 
					Without understanding how to program one cannot expect to understand how someone else did it without even having the source code. 
				\item[Assembly] 
					Due to the compilation process, one often cannot gain access to the source code of the program. 
					While call flow graphs are useful, to gain a true understanding of what the program is doing, one needs to understand Assembly code. 
				\item[Side channel attacks] 
					This is attacking a program through alternative means. 
					Rather than attempting to break it through direct reverse engineering, it uses dependencies or some other means. 
				\item[ltrace and strace] 
					These are library and system call tracers. 
					They are useful in determining how a program interacts with the system and how to attack it. 
					They can also be useful in determining how to conduct side channel attacks. 
			\end{description}
		\subsection{Web}
			This section requires that the member attack a number of websites using common vulnerabilities in them. 
			This is one of the most common attacks in the modern world. 

			Through an analysis of post challenge ``write ups''\footnote{\url{https://github.com/ctfs/write-ups-2016/tree/master/boston-key-party-2016/web}}
			and completing some of the individual tasks, 
			it was determined that the following skills and technical concepts are required to conduct the task: 
			\textit{Content Security Policy, HTTP Redirect Attacks, Network Listeners, HTTP Directory Traversal, HTTP Directory Creation, Base64 Encoding} and \textit{SQL Injection}.
			Each of these is given in more detail below. 
			\begin{description}
				\item[Content Security Policy]
					This is a standard introduced to prevent common web attacks. 
					It is based on approving the origin of changes or additions to a website. 
					However, like every security mechanism, it can be circumvented if poorly implemented. 
				\item[HTTP Redirect attacks] 
					This is based on systems that will redirect based on a parameter given in the request. 
					It is commonly used in phishing attacks. 
				\item[Network listeners] 
					Programs such as ``nc -l'' or other network listeners can be exceptionally useful in web based attacks to determine whether something it attempting to return to your system. 
				\item[HTTP Directory Traversal] 
					Poorly configured servers will allow you to use the POSIX ``..'' link within a directory to move about. 
					This means that you can gain access to parts of the system that the server was set to block. 
				\item[HTTP/Ruby Directory Creation] 
					Some systems will also allow for the creation of directories. 
					This can be used to alter the systems running and as a step for gaining a higher level of access, especially when coupled with directory traversal. 
				\item[Base64 encoding] 
					This is a method of storing and transmitting data in a different format. 
					It looks similar to a hash, but often contains padding equals signs. 
					If data is stored in this format, it is trivial to return it to a readable form. 
				\item[SQL injection through encoding changes] 
					SQLi is a common form of database based web attack. 
					However, it is also often setup in such a way that the data that is entered is filtered. 
					Encodings will sometimes allow you to bypass these filters and insert your SQL code into the database. 
			\end{description}

	\section{Breakdown of Requirements}
	 	The following is a breakdown of how many times a particular area was used in the challenge. 
		These areas are broken into Major Categories (in bold) and minor categories, with a count for how many times each arose. 
		There is a weighting section which conveys how important a given section is to completing the whole CTF event. 
		\begin{table}[H]
			\centering
			\begin{tabular}{| l | l | l |}
				\hline
				\textbf{Category} & \textbf{Number} & \textbf{Weight} \\ \hline 
				\textbf{Cryptography} & \textbf{8} & \textbf{0.8}\\ \hline
				\quad SHA Hashing & 1 & \\ \hline 
				\quad RSA Key Recovery & 1 & \\ \hline
				\quad Eilliptic Curve Encryption and Attacks & 1 & \\ \hline
				\quad Signatures & 1 & \\ \hline 
				\quad Common RSA Attacks & 1 & \\ \hline
				\quad DES Weak Keys & 1 & \\ \hline 
				\quad Hash Collisions & 1 & \\ \hline 
				\quad Base64 Encoding & 1 & \\ \hline 
				\textbf{Web} & \textbf{6} & \textbf{1} \\ \hline 
				\quad Content Security Policy & 1 & \\ \hline 
				\quad HTTP Redirect attacks (XSS) & 1 & \\ \hline 
				\quad Network Listeners & 1 & \\ \hline 
				\quad HTTP Directory Traversal & 1 & \\ \hline 
				\quad HTTP/Ruby Directory Creation & 1 & \\ \hline 
				\quad SQL Injection through character set encoding & 1 & \\ \hline 
				\textbf{Exploitation} & \textbf{5} & \textbf{0.7} \\ \hline 
				\quad RIP Redirection & 1 & \\ \hline 
				\quad Common C Flaws & 1 & \\ \hline 
				\quad Smashed Stack Cleaning & 1 & \\ \hline 
				\quad Heap Overflows & 1 & \\ \hline 
				\quad libc Attacks & 1 & \\ \hline 
				\textbf{Reverse Engineering} & \textbf{3} & \textbf{0.6} \\ \hline
				\quad Tools (IDA/Radare2) & 1 & \\ \hline 
				\quad Assembly & 1 & \\ \hline 
				\quad Side Channel Attacks & 1 & \\ \hline 
				\quad ltrace & 1 & \\ \hline 
				\textbf{Programming} & \textbf{3} & \textbf{1}\\ \hline
			\end{tabular}
			\caption{Breakdown of Boston Key Party CTF Topics}
			\label{tab:BCTFBreakdown}
		\end{table}
		Using both the weighting and the count columns, one can determine the most useful content to place into the training program. 
		This will be used in determining how much depth will be entered into when writing the training program. 	
\chapter{Plaid CTF 2015}
	\section{Overview of Challenge}
		Plaid CTF\footnote{\url{https://play.plaidctf.com/}} is the annual CTF run by the \href{http://pwning.net/}{``Plaid Parliament of Pwning''}.
		While lower profile in Australia, this CTF is both international and set at a far higher difficulty level than CySCA. 
		It's timing within the year allows just enough time for third years returning to the VECC to rebuild their skills after leave and attempt the challenge. 
		This makes it a good CTF to use for ensuring that the higher level skills have been covered by the training program. 

		This CTF is also a DEFCON qualifier. 
		This means that it is held at the same high standard as the Boston Key Party, while also having a broader range of challenges and being harder overall. 

		The team that runs this CTF is from Carnegie Mellon University and runs a number of other CTFs, such as picoctf. 
		This team is regularly one of the highest scoring teams on CTF Time, making them experts in the competitions. 
		Due to this, their own CTF competitions are often well built and difficult. 
	
	\section{Main Areas Of Focus}
		This section will explain the skills required for the completion of the challenges within the CTF. 
		It will list the requirements for each section, defining them so that they can be implemented into the training program. 
		The sections within this CTF are:\footnote{\url{https://ctftime.org/event/185}}
		\subsection{Cryptography}
			This section has the member determine weaknesses in the implementation of cryptographic algorithms. 
			Each challenge requires an understanding of a certain type of encryption as well as its possible failures. 


			Through an analysis of post challenge ``write ups''\footnote{\url{https://github.com/ctfs/write-ups-2015/tree/master/plaidctf-2015/crypto}}
			and completing some of the individual tasks, 
			it was determined that the following skills and technical concepts are required to conduct the task: 
			\textit{RSA, Programming, Data Representation, Server Authentication} and the \textit{Merkle-Hallman Knapsack Cryptosystem}. 
			Each of these is given in more detail below. 
			\begin{description}
				\item[Understanding of RSA Encription]
				\begin{itemize}
					\item Understand the \{N : e : c\} format. 
					\item Understand the common requirements for e and d in RSA. 
				\end{itemize}
				\item[Programming] 
					The act of telling a computer what to do and how to do it. 
				\item[Data Representation] 
					Data can be stored, interacted with and represented in many different ways. 
					Understanding this can be a key factor in breaking into a system. 
				\item[Server Authentication]
					Servers use a number of systems, such as private/public key pairs and cookies to authenticate. 
					Understanding this can allow an attacker to find weaknesses in the authentication system. 
				\item[Merkle-Hallman Knapsack Cryptosystem]
					A public/private key system similar to RSA, but significantly weaker. 
					A system using this can be broken using known tools. 
			\end{description}
		\subsection{Forensics}
			This section has the member attempt to determine the types and contents of unknown files. 
			To successfully complete it, the member will need to have a good understanding of error checking and file types. 

			Through an analysis of post challenge ``write ups''\footnote{\url{https://github.com/ctfs/write-ups-2015/tree/master/plaidctf-2015/forensics}}
			and completing some of the individual tasks, 
			it was determined that the following skills and technical concepts are required to conduct the task: 
			\textit{File Types, File Conversions, File Specifications, DOS -> UNIX conversions, Cyclic Redundancy Checks} and \textit{Programming}.
			Each of these is given in more detail below. 
			\begin{description}
				\item[Understanding of file types] 
					The file types and their heads/trailers can be used to determine whether multiple files are stored in one, or why a file is corrupted. 
					This understanding allows one to repair or retrieve corrupted files.
				\item[File conversions] 
					This is the process of taking one file and creating another type of file from it. 
					It often has some artifacts which are left and can be used to gain information about the system. 
				\item[File Specifications]
					This is the standard headers and trailers as well as the IVs which make up a file. 
					Understanding this can allow an attacker to understand what the file is that they are finding on a bare disk and how to read it. 
				\item[DOS -> Unix (CRLF -> LF) conversion]
					This conversion can cause issues with some files. 
					It occurs due to DOS using the full ``\verb+\r\n+'' line endings, while Unix uses only ``\verb+\n+''. 
				\item[Cyclic Redundancy Codes]
					These are used to check whether a file has been altered while stored or in transit. 
					They can then be used to return the file to it's original state. 
				\item[Programming]
					This is the act of telling a computer how to do something. 
					It is useful here as automation of slow or arduous tasks. 
			\end{description}

		\subsection{Pwnable}
			This section requires the crafting of shellcode and manual exploit discovery. 
			The member will discover the flag only when they have gained access to a shell on the machine that the service was running on. 

			Through an analysis of post challenge ``write ups''\footnote{\url{https://github.com/ctfs/write-ups-2015/tree/master/plaidctf-2015/pwnable}}
			and completing some of the individual tasks, 
			it was determined that the following skills and technical concepts are required to conduct the task: 
			\textit{Shellcoding, x86\_64 Assembly, Memory Allocation, GDB, ARM Assembly, ELF Files, C Exploitation, URL Injection, HTTP, HTTP Directory Traversal} and \textit{Local File Inclusion}
			Each of these is given in more detail below. 
			\begin{description}
				\item[Shellcoding]
					This is the act of writing code which will give an attacker a shell on another system. 
					It is usually done in assembly. 
				\item[x86\_64Registers and Assembly] 
					This is the base programming language of most computers. 
					It is assembled into machine code, which can then be run by the machine. 
					It is also the basis for shell coding. 
				\item[Memory allocation] 
					Memory allocation and it's standards is one of the key parts of writing shellcode. 
					This is due to the fact that you will have only a small amount of memory in which to write the code and its data. 
					Understanding how this works allow one to create shell code that will work in most systems, as well as injecting shellcode into a program in the first place. 
				\item[GDB (and Peda)]
					This is a tool used for exploit development. 
					It allows one to step through a program, seeing the full memory and register setup of the computer as it is processed. 
				\item[ARM assembly]
					Like x86 assembly, this is the base machine language for ARM processors. 
				\item[ELF files]
					These are the standard Linux executable files. 
					Understanding how they work will allow easier analysis of them. 
				\item[C Exploitation]
					C has a number of pitfalls that programmers commonly fall into. 
					Understanding these pitfalls allows an attacker to look for the most common avenues of attack. 
				\item[URL Injection]
					Knowing what to look for when attempting to gain further access to a website is helpful to the attack. 
					Understanding URL injection is useful to gaining access that you are not expected to have. 
				\item[HTTP] 
					This is the main protocol used to communicate between computers and web servers. 
					Understanding it will allow you to exploit common flaws in the implementation of it, or the programs using it. 
				\item[Web Directory Traversal] 
					This is a common attack which will allow one to access part of a web server that they did not have permissions for. 
					It can be used for gathering further data on the server, or for editing files and gaining full access. 
				\item[Local File Inclusion]
					This is the precursor to a number of attacks, such as arbitrary code execution or reflected XSS. 
					Understanding this can be a major step to either gaining access to a server, or gaining access to a users machine. 
			\end{description}
		\subsection{Reverse Engineering}
			This section has a member attempt to understand how a program is working in order to exploit it. 
			The member will be required to use the tools at their disposal to determine algorithms and computational flaws in the provided programs. 

			Through an analysis of post challenge ``write ups''\footnote{\url{https://github.com/ctfs/write-ups-2015/tree/master/plaidctf-2015/reversing}}
			and completing some of the individual tasks, 
			it was determined that the following skills and technical concepts are required to conduct the task: 
			\textit{Radare2, IDA64, Rotational Algorithms, Algorithmic Flaws, Regex} and \textit{Boolean Mathematics}.
			Each of these is given in more detail below. 
			\begin{description}
				\item[Reverse Engineering tools] 
					These are tools which allow one to take a compiled binary and follow it's assembly code and system calls to determine what it does. 
					The following are the two main tools in the area:
					\begin{itemize}
						\item radare2
						\item IDA 64
					\end{itemize}
				\item[Rotational Algorithms]
					This is a common part of many cryptographic standards such as AES. 
					Understanding how it works will allow an attacker to find flaws in harder cryptosystems.
				\item[Algorithmic flaws] 
					Understanding common algorithmic flaws is useful in determining how to break an algorithm as well as how to guess what it's output will be. 
					This can be useful when attempting break into a system which is protected by a proprietary authentication system. 
				\item[Regex]
					A common method of finding data in a large collection. 
					Used for specifying a generic schema and collecting all matching items. 
				\item[Boolean Mathematics] 
					The logic which bounds all computer processing and programming. 
					It deals with truthfulness and comparison. 
			\end{description}
	
	\section{Breakdown of Requirements}
	 	The following is a breakdown of how many times a particular area was used in the challenge. 
		These areas are broken into Major Categories (in bold) and minor categories, with a count for how many times each arose. 
		There is a weighting section which conveys how important a given section is to completing the whole CTF event. 

		\begin{center}
			\begin{longtable}{| l | l | l |}
				\hline
				\textbf{Category} & \textbf{Number} & \textbf{Weight} \\ \hline 
				\endhead
				\multicolumn{3}{|r|}{{Continued on next page}} \\ \hline
				\endfoot
				\endlastfoot
				%---------------------------------------------------------------
				\textbf{General} & \textbf{6} & \textbf{0.5} \\ \hline
				\quad Data Representation & 1 & \\ \hline
				\quad ELF Files & 1 & \\ \hline
				\quad Rotational Ciphers & 1 & \\ \hline
				\quad Algorithmic Flaws & 1 & \\ \hline 
				\quad Regex & 1 & \\ \hline 
				\quad Boolean Mathematics & 1 & \\ \hline 
				\textbf{Exploitation} & \textbf{6} & \textbf{0.7}\\ \hline
				\quad Shellcode & 1 & \\ \hline
				\quad x86\_64 Registers and Assembly & 1 & \\ \hline
				\quad Memory Allocation & 1 & \\ \hline
				\quad GDB and Peda & 1 & \\ \hline
				\quad ARM Assembly & 1 & \\ \hline
				\quad C Exploitation & 1 & \\ \hline 
				\textbf{Cryptography} & \textbf{5} & \textbf{0.8} \\ \hline
				\quad RSA Encryption & 3 & \\ \hline
				\qquad \{N : e : c\} Format & 1 & \\ \hline
				\qquad Requirements for ``e'' and ``d'' in RSA & 1 & \\ \hline
				\quad Server Authentication & 1 & \\ \hline
				\quad Merkle-Hallman Knapsack Cryptosystem & 1 & \\ \hline
				\textbf{Web Exploitation} & \textbf{4} & \textbf{1} \\ \hline
				\quad URL Injection & 1 & \\ \hline
				\quad HTTP & 1 & \\ \hline 
				\quad Directory Traversal & 1 & \\ \hline
				\quad Local File Inclusion & 1 & \\ \hline 
				\textbf{Forensics} & \textbf{4} & \textbf{0.7} \\ \hline
				\quad File Types & 1 & \\ \hline
				\quad File Conversions & 1 & \\ \hline 
				\quad Dos -> Unix (CRLF -> LF) conversion & 1 & \\ \hline 
				\quad Cyclic Redundancy Checks & 1 & \\ \hline
				\textbf{Reverse Engineering} & \textbf{2} & \textbf{0.6} \\ \hline
				\quad Radare2 & 1 & \\ \hline
				\quad IDA 64 & 1 & \\ \hline 
				\textbf{Programming} & \textbf{2} & \textbf{1} \\ \hline
				\noalign{\vskip 0.5cm}
				\caption{B\label{tab:PlaidCTF Breakdown}Breakdown of PlaidCTF Topics}
				\vspace{-1.3cm}
			\end{longtable}
		\end{center}
		Using both the weighting and the count columns, one can determine the most useful content to place into the training program. 
		This will be used in determining how much depth will be entered into when writing the training program.

\chapter{Summary}
	\textit{CySCA}, \textit{the Boston Key Party} and \textit{PlaidCTF} cover a wide range of topics due to their differing goals. 
	CySCA covers the broadest range, but at the lowest level. 
	This makes it well suited as the goal of the CYSEC VECC at the current time due to the low overall level of skill. 
	However, CTFs such as the Boston Key Party and PlaidCTF have also been analysed for use when the VECC gains a higher level of skill. 
	These two CTFs, in the order that they are presented, are good higher level challenges for next year, when there are a number of members who have some experience with the activity. 

	Included in table \ref{tab:summary} is the summary breakdown of all of these events. 
	This shows the skills required to complete all of the tasks required for all of these CTFs, as well as the number of times a particular skill comes up and the weighting given to the main category based on it's overall importance to completing the challenge. 

			\begin{center}
				\begin{longtable}{| l | l | l |}
					\hline
					\textbf{Category} & \textbf{Number} & \textbf{Weight} \\ \hline 
					\endhead
					\multicolumn{3}{|r|}{{Continued on next page}} \\ \hline
					\endfoot
					\endlastfoot
					%--------------------------------------------------------------
					\textbf{Web Penetration} & \textbf{14} & \textbf{1} \\ \hline
					\quad HTTP Directory Traversal & 3 & \\ \hline
					\quad HTTP MITM & 2 & \\ \hline
					\quad URL Injection & 2 & \\ \hline
					\quad HTTP/ruby Directory Creation & 1 & \\ \hline 
					\quad Session Hijacking & 1 & \\ \hline
					\quad Local File Inclusion & 1 & \\ \hline 
					\quad Content Security Policy & 1 & \\ \hline 
					\quad HTTP Redirect Attacks & 1 & \\ \hline 
					\quad Network Listeners & 1 & \\ \hline 
					\quad SQLi through Character Encoding & 1 & \\ \hline 
					\newpage
					\textbf{Programming} & \textbf{13} & \textbf{1} \\ \hline
					\quad Python & 3 & \\ \hline 
						\qquad Shellcode & 1 & \\ \hline 
						\qquad Builtins & 1 & \\ \hline 
						\qquad Pickled input & 1 & \\ \hline 
					\quad Sockets & 2 & \\ \hline 
					\quad JavaScript & 1 & \\ \hline 
					\quad Obfuscation & 1 & \\ \hline
					\textbf{General Knowledge} & \textbf{12} & \textbf{0.5}\\ \hline 
					\quad CVE and CVE attack Matching & 1 & \\ \hline 
					\quad Data Representation & 1 & \\ \hline
					\quad ELF Files & 1 & \\ \hline
					\quad Rotational Ciphers & 1 & \\ \hline
					\quad Algorithmic Flaws & 1 & \\ \hline 
					\quad Regex & 1 & \\ \hline 
					\quad Boolean Mathematics & 1 & \\ \hline 
					\quad Binary Operations & 1 & \\ \hline 
					\quad Logic & 1 & \\ \hline 
					\quad Trivia, General knowledge & 1 & \\ \hline 
					\quad Mathematics & 1 & \\ \hline 
					\quad QR Codes & 1 & \\ \hline 
					\textbf{Cryptography} & \textbf{12} & \textbf{0.8} \\ \hline 
					\quad RSA Encryption & 3 & \\ \hline
						\qquad \{N : e : c\} Format & 1 & \\ \hline
						\qquad Requirements for ``e'' and ``d'' in RSA & 1 & \\ \hline
						\qquad Common RSA Attacks & 1 & \\ \hline 
					\quad Hash Cracking & 2 & \\ \hline
						\qquad SHA & 1 & \\ \hline 
						\qquad Collisions & 1 & \\ \hline 
					\quad Eilliptic Curve Encryption and Attacks & 1 & \\ \hline 
					\quad Signatures & 1 & \\ \hline 
					\quad DES Weak Keys & 1 & \\ \hline 
					\quad Server Authentication & 1 & \\ \hline
					\quad Merkle-Hallman Knapsack Cryptosystem & 1 & \\ \hline
					\quad Cryptographic Flaws & 1 & \\ \hline 
					\quad Encoding & 1 & \\ \hline 
					\newpage
					\textbf{Forensics} & \textbf{11} & \textbf{0.7}\\ \hline
					\quad PCAP & 2 & \\ \hline
					\quad File Types & 1 & \\ \hline
					\quad File Conversions & 1 & \\ \hline 
					\quad Dos -> Unix (CRLF -> LF) conversion & 1 & \\ \hline 
					\quad Cyclic Redundancy Checks & 1 & \\ \hline
					\quad Network Forensics & 2 & \\ \hline 
					\quad DOXXING & 1 & \\ \hline 
					\quad Attack Traffic & 1 & \\ \hline 
					\quad Disk and Memory Analysis & 1 & \\ \hline 
					\textbf{Exploitation} & \textbf{10} & \textbf{0.7} \\ \hline
					\quad C Exploitation & 2 & \\ \hline 
					\quad x86\_64 Registers and Assembly & 2 & \\ \hline
					\quad Shellcode & 1 & \\ \hline
					\quad Memory Allocation & 1 & \\ \hline
					\quad GDB and Peda & 1 & \\ \hline
					\quad ARM Assembly & 1 & \\ \hline
					\quad RIP Redirection & 1 & \\ \hline 
					\quad Smashed Stack Cleaning & 1 & \\ \hline 
					\quad Heap Overflows & 1 & \\ \hline 
					\quad libc Attacks & 1 & \\ \hline 
					\textbf{Reverse Engineering} & \textbf{10} & \textbf{0.6}\\ \hline 
					\quad Radare2 & 2 & \\ \hline 
					\quad IDA 64 & 2 & \\ \hline 
					\quad C\# Reverse Engineering & 1 & \\ \hline
					\quad Blind Reverse Engineering & 1 & \\ \hline 
					\quad Code Testing & 1 & \\ \hline 
					\quad Side Channel Attacks & 1 & \\ \hline 
					\quad ltrace & 1 & \\ \hline 
					\quad Malware Analysis & 1 & \\ \hline
					\textbf{Network Exploitation} & \textbf{4} & \textbf{0.8} \\ \hline 
					\quad Attack Persistence & 1 & \\ \hline 
					\quad Metasploit & 1 & \\ \hline 
					\quad Nmap & 1 & \\ \hline 
					\quad Heartbleed & 1 & \\ \hline
					\textbf{Network Defence} & \textbf{3} & \textbf{0.6} \\ \hline
					\quad STIX Reporting & 2 & \\ \hline
					\quad Snort & 1 & \\ \hline
					\newpage
					\textbf{Networking} & \textbf{2} & \textbf{0.8}\\ \hline
					\quad DNS Enumeration and Zone Transfer & 1 & \\ \hline 
					\quad HTTP & 1 & \\ \hline 
					\textbf{Linux CLI} & \textbf{1} &\textbf{0.6} \\ \hline
					\quad Command Injection & 1 & \\ \hline 
					\noalign{\vskip 0.5cm}
					\caption{\label{tab:summary}Summary of CTF Skills}
					\vspace{-1.3cm}
				\end{longtable}
			\end{center}
	
	The Skills included in table \ref{tab:summary} are the basis for a good understanding of cyber security as seen in CTFs. 
	However, these events are not the only part of the industry. 
	These skills have been weighted to bring out the necessity of the skill not only to the CTF competitions listed here, 
	but also to a member entering the industry. 
	Thus, when compiling the training program, one should take these weightings into account. 
	Items weighted higher should be taught with more depth and focus, while items weighted lower should have less focus and time. 

	The skills outlined here also include a number of specific tools. 
	While these tools don't constantly change, they do evolve and occasionally become deprecated. 
	Due to this, any training should focus on skills, rather than tools. 
	This will enable those being taught to learn the necessary tools in their own time, 
	gaining the skill necessary to learn new ones as they are released. 

	It is also worth noting that while these CTFs are a good source of skills information and a good means of testing, 
	they have little worth as a first principles teaching tool. 
	This occurs due to the nature of CTFs. 
	They are designed to test the highest levels of students in the field of cyber security, ensuring that they cover as many aspects of the field as possible. 
	Furthermore, for the same reason, they are often pitched far higher than a new student would be able to understand. 

	However, while these challenges are not useful for teaching in themselves, it is evident that the excel at testing one's skill in the area. 
	This is the reason that they are commonly used for recruiting---the main purpose of CySCA---they quickly separate those with skills in the field from those who do not have them. 

	From this document, the contents of a training program can be created. 
	Each row of table \ref{tab:summary} will become a section within the program with an overview of content as well as links to other resources available. 
	This will be used to form the instructors notes for weekly lessons on these skills, stepping students through their use and giving them the tools to explore it themselves. 
\end{document}
