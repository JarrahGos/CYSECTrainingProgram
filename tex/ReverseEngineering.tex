\chapter{Reverse Engineering}
	\label{ch:ReverseEngineering}
	Reverse engineering is the process of taking a fully compiled (and often obfuscated) binary and working out how it works. 
	This process usually requires understanding programming, algorithms and assembly, but can occasionally be made easier by language constructs such as Java's bytecode, which allows full decompilation. 
	This chapter will discuss first the language used for the majority of reverse engineering, Assembly. 
	Next, it will discuss the main tools used in the process. 
	Finally, corner cases such as Java reverse engineering will be discussed, along with an example project. 

	\section{Assembly}
		Assembly is the base language of the given computer. 
		It can be used to write any program, but often is far to inefficient for the programmer to use. 
		Originally, this was the only way to program a computer. 
		However, those working in computer science quickly created higher level languages. 
		Thus, it has been the domain of ``Hackers'' and reverse engineers since the mid-1950's. 

		This section will discuss the basics of both x86 (Intel) and ARM assembly. 
		These are currently the two most common versions of the code, each being used on PCs and mobile devices respectively. 
		\subsection{x86 Assembly}

		%TODO: Decide whether this goes here or in programming. 
		\subsection{ARM Assembly}
	\section{Radare2}
		Radare2 is a set of tools for working with binary files. 
		This makes it useful as a means of reading and altering compiled programs and their data files from the command line. 
		While the Radare2 suite has a number of stand alone programs within it, this section will cover only the main program. 
		This program allows for the searching, altering and disassembly of binary programs in multiple modes, making it a great tool for challenges such as Pwn Adventure in the next section.
		For a more detailed guide, read through the Radare2 book.\cite{Radare2}

		\subsection{Getting Started}
			The learning curve at the start of using Radare2 is quite steep. 
			However, if you have any experience with other command line tools such as Vim and GDB, many of the commands and shortcuts should come easily. 
			Before going any further, it is worthwhile to note that you can get the help information for any command by appending a ``?'' to it. 

			Navigation within Radare2 is done through the following commands:
			\begin{description}
				\item[Seek]
					or ``s'' will move you to the address that you specify after the command. 
					The address can either be an explicit memory location, or a mathematical statement based on the current location or a label within the binary. 
					An example of this is ``s 0x00001c80''.
				\item[is]
					or information symbols will get you a list of the different symbols which are still in the binary. 
					These can be functions, objects, strings or any other symbol that survives the compilation process. 
					This is useful to find the functions which the program uses, and will become the main way of navigating in the Pwn Adventure Challenge. 
				\item[Visual mode]
					Like Vim, Radare2 has a visual mode which allows for direct reading and writing of assembly or hex. 
					In this mode, you can move using the same key bindings as Vim, with ``i'' being used as replace and ``p'' being used to change visual mode. 
					Furthermore, to get a cursor which can be used for these insert functions, use ``c''.
					This mode allows for the editing of the disassembled binary without having to reassemble it. 
				\item[Hex View]
					In visual hex view, hold shift while moving to highlight a byte range. 
					Tab will switch from the Hex pane to the ASCII pane. 
				\item[\~{}]
					This is the internal grep function. 
					When it is placed after a command, anything after it will be used to filter the results. 
					Furthermore, output can be filtered by column using the syntax ``[x]'' or by row using ``:x'' after the grep search.
			\end{description}
			Beyond this, you will want to start the program with the ``-w'' flag in order to write to the file. 
			Finally, any CLI program---or syntax such as pipes and redirection---installed on your computer can be accessed through the Radare2 prompt. 
			This is useful for actions such as greping through the output of ``is''. 

			At this point you should have enough of an understanding of Radare2 to be able to use it. 
			The next section contains spoilers to the challenges presented by Pwn Adventure 3. 
			Before reading on, recommend downloading and installing the game and attempting to reverse engineer it so that you can win. 
			I recommend starting with the sprint speed. 
			%TODO: Should there be more than this section here?
	\section{IDA 64}
	\section{Other Tools}
		\subsection{Strings}
		\subsection{Ltrace}
		\subsection{Strace}
	\section{Side Channel Attacks}
	\section{Java Reverse Engineering}
		Unlike most fully complied languages, Java is quite easy to fully decompile, rather than just disassemble. 
		This means that you will be able to get source code, though not necessarily the same source code as the programmer wrote in the first place. 
		Thus, Java reverse engineering is simply a matter of choosing the best program to use in decompilation. 

		The current recommendation for Java Decompilers is Procyon\footnote{\url{https://bitbucket.org/mstrobel/procyon}}
		This is due to the fact that it is the most up to date and well maintained decompiler available.\footnote{You may hear of others such as jad. This is old and unmaintained.} 
		
		Procyon is quite easy to use. 
		All that is required is having Java installed on the system that you are working on and running the provided JAR package. 
		You will give it both the class files that you wish to decompile, as well as an output directory. 
		An example of this can be found below:
		\begin{lstlisting}[style=CLI]
			$ ./procyon.jar *.class -o src
		\end{lstlisting}

		Furthermore, Procyon will allow you to decompile a JAR file in it's entirety. 
		All that need be done for this is running the same command as above on the JAR file rather than the individual unpacked classes. 

		The reason that Java Reverse engineering is seen as easy is the fact that you now have the full source code of the program. 
		At worse, the programmer may have obfuscated the code, making it harder to determine what the methods and variables mean. 
		This would mean that you will have to manually determine this from the usage of the variables and the content of the methods.

		Once this decompilation has been completed, you can use tools such as yFiles to get an overview of the classes or data structures. 
		Other than this, it is simply a matter of finding the logic within the program that is used to calculate the value that you are reverse engineering for. 
	
		%TODO: Revise this section
	\section{Pwn Adventure}
