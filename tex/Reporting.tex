\chapter{Reporting}
	\label{ch:Reporting}
	While at this point you may be able to access most networks unauthorised, the communication of this has not been covered. 
	This skill is one of the most important if one is to see the state of the network improve in any way. 
	As such, this chapter will discuss both the process of writing a report on a test that has been conducted 
	and the current standard for passing incident and vulnerability data to another organisation or program. 

	\section{Report Writing}
		While far from being the most glamerous or interesting part, writing the report is one of the most important parts of a penetration test\cite{playbook}. 
		It is the part of the test that will ensure vulnerabilities are removed and your work will continue. 
		Furthermore, it is a statement of your own skill in expressing what you did and how it effects the network being attacked.

		The easy solution here is to simply repackage the vulnerability scans tat were conducted. 
		This will give a report listing potential vulnerabilities in the network and their potential serverity. 
		However, this tells the reader nothing about the data that could be accessed through the vulnerabilities, 
		nor anything about the process and difficulty of exploiting them. 
		Thus, such a report seems cheap and useless to the customer, making them less likely to implement anything you advise. 

		Alternatively, a report could be detailed in how the exploit was discovered, exploited and what was exfiltrated. 
		However, there is a point where this becomes too much, necessitating creating a report that is split between 
		``technical'' and ``non-technical'' readers. 
		Such a report is likely to have more sway with both the companies executive staff and their system administrators. 
		Vulnerabilities that have been found, exploited and used to exfiltrate data will be fixed over those with no evidence behind them. 

		The general rules for report writing are as follows:
		\begin{itemize}
			\item Under no circumstances submit a reformatted vulnerability scanner report. 
				This gives no information useful to the company that they couldn't have used the scanner for, making your role as a penetration tester redundant. 
			\item Rate vulnerabilities consistantly, both by their natural severity and the data possible to exfiltrate using them. 
			\item Explain the effects of a given vulnerability on the CIA Triad. 
			\item Use vulnerabilty scanner output for determining serverty, but do not rely on it. 
			\item Discuss exploits that you have successfully tested differently to those that you have not. 
			\item For each finding that you discover, provide a solution to it. 
				At a minimum, provide a mitigation strategy. 
			\item Validate that findings provided by scanners are not based purely on versions, but rather are expoitable on the given machine. 
			\item Create a standard template (such as a \LaTeX{} template and class) and follow it for all reports. 
				This will ensure both formatting and content is consistent accross reports. 
		\end{itemize}

		As a guide line, the following is a section guide for a basic report:
		\begin{enumerate}
			\item \textbf{Introduction:} The overview and top level description of the document. 
				This section should contain an executive summary and 
				(if you are splitting the document for techinical and non-technical readers)
				a reading guide outlining which sections should be read by whome. 
			\item \textbf{Scope:} This section details what limitations were given, what was tested and why. 
				It should give IP ranges, names and URLs of major targets, as well as the objective of hitting those targets. 
			\item \textbf{Changes to requirements:} An overview of changes to the original test paramiters given, if any. 
			\item \textbf{Methodology:} How you went about information gathering, targeting, exploitation and post-exploitation. 
			\item \textbf{Significant findings:} Critical findings only here, with a basic outline of why it is severe 
				and what it gives access to. 
				Expect this to be read by executives rather than technical workers. 
			\item \textbf{Posative Observations:} This is the parts of security that the organisation did well. 
				It should be used to provide a posative note in a report that may otherwise be negative and overbearing. 
				Furthermore, it should be used to reward good practices, as they may only exist on a short teather. 
			\item \textbf{Findings Summary:} An overview of all vulnerabilities found based on severity and exploitability. 
				Break it further into the following sections:
				\begin{itemize}
					\item Practical Exploits: vulnerabilities that you have been able to take through the whole exploitation cycle and exfiltrate data from. 
					\item Theoratical Exploits: vulnerabilities that you have not been able to use in the test. 
				\end{itemize}
				Again, this section should be an overview for the more studious exeuctives and IT managers, rather than detailed content written for sysadmins or security professionals. 
			\item \textbf{Detailed Findings:} Following the same structure as above, go into detail about what each vulnerability
				is, how it was expoited and why it was given the severity it was. 
				This is the place to go into technical detail, explaining exactly what was done and how to fix it. 
				This is also the section in which you should place relevant short logs, captured data and other details. 
			\item \textbf{Appendix:} This would have scans, exfiltrated data and any aditional information that the reader doesn't want to sift through in the main document. 
				However, if you put it in here, it must be referenced within the main text, 
				this section is not a dumping ground to make the document look bigger. 
		\end{enumerate}
		
		In addition to this, you may want to report on Open Source Intelligence that was used to enumerate the location and 
		data of the organisation. 
		This will be usefull if they are attempting to reduce their online presence, or simply to know where attacks are more likely to hit. 
		Furthermore, you should provide a means of checking whether mitigations they may put in place were successful. 
		This may be through offering your services in a limited fashion to test again. 
		However, you may also write scripts which can be executed in a specified environment to test the vulnerabilty automatically. 

	\section{STIX Model}
		\url{http://stixproject.github.io/}
\newpage
